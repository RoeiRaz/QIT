\documentclass[a4paper,10pt]{hw}
\usepackage[utf8]{inputenc}
\usepackage{amsmath,amsthm,amssymb}
\usepackage{mathtools}

% Bra-Ket hack from stack exchange
\DeclarePairedDelimiter\bra{\langle}{\rvert}
\DeclarePairedDelimiter\ket{\lvert}{\rangle}
\DeclarePairedDelimiterX\braket[2]{\langle}{\rangle}{#1 \delimsize\vert #2}

%opening
\title{Introduction to Quantum Information Processing}
\author{Roei Rosenzweig 313590937,\\ Roey Maor 820082000}

\begin{document}

\maketitle

\section{}

\subsection{}

$tr(A^{\dagger}) = \sum_{i=1}^{n}{a_{ii}^*} = (\sum_{i=1}^{n}{a_{ii}})^* = (tr(A))^*$

\subsection{}

$tr(\alpha A + \beta B) = \sum_{i=1}^{n}{\alpha A_{ii} + \beta B_{ii}} = \alpha\sum_{i=1}^{n}{A_{ii}} + \beta\sum_{i=1}^{n}{B_{ii}} = \alpha tr(A) + \beta tr(B)$

\subsection{}

$tr(AB) = \sum_{i=1}^{n}{(AB)_{ii}} = \sum_{i=1}^{n}{(\sum_{k=1}^{n}{A_{ik} B_{ki}})} = \sum_{i=1}^{n}{(\sum_{k=1}^{n}{B_{ki} A_{ik}})} = \sum_{i=1}^{n}{(BA)_{ii}} = tr(BA)$ 

\section{}

Throughout this part we mark eigenvalues with $\lambda_i$ and eigenvectors with $v_i$.
In order to find the eigenvalue we use the identities $tr(A) = \sum{\lambda_i}$ and $det(A) = \prod{\lambda_i}$.

\subsection{}

$$\begin{array}{lclcl}
   tr(H) & = & \lambda_{1} + \lambda_{2} & = & 0\\
   det(H) & = & \lambda_{1} \lambda_{2} & = & -1
\end{array}$$

Solving for $\lambda_{1,2}$ yields:

$$\begin{array}{lcr}
   \lambda_1 & = & \pm1 \\
   \lambda_2 & = & \mp1
\end{array}$$

Because of the symmetry, we can just assume $\lambda_1 = 1$ and $\lambda_2 = -1$.
To find the eigenvectors we solve for $v_i$:
$$\begin{array}{lcr}
   (H - I\lambda_i) v_i = 0
  \end{array}$$
  
TODO

\section{}

\subsection{}

\subsubsection{}

$\sigma_x$ is a symmetric and real matrix, and as such it is also Hermitian.

\subsubsection{}

$$\sigma_y^\dagger
= \begin{pmatrix} 0 & -i \\ i & 0 \end{pmatrix}^\dagger
= \begin{pmatrix} 0 & i^* \\ -i^* & 0 \end{pmatrix}
= \begin{pmatrix} 0 & -i \\ i & 0 \end{pmatrix}
= \sigma_y$$

because $\sigma_y^\dagger = \sigma_y$, $\sigma_y$ is Hermitian.

\subsubsection{}

$\sigma_z$ is a symmetric and real matrix, and as such it is also Hermitian.

\subsubsection{}

$H^\dagger = \frac{1}{\sqrt{2}}^\star \cdot \begin{pmatrix} 1 & 1 \\ 1 & -1 \end{pmatrix}^\dagger \stackrel{\text{real numbers, symmetric matrix}}{=} \frac{1}{\sqrt{2}}
\cdot \begin{pmatrix} 1 & 1 \\ 1 & -1 \end{pmatrix} = H$

\subsection{}

\subsubsection{}

\begin{align*}
\left(\alpha A + \beta B\right)^\dagger &
\\ &= \left(\alpha A\right)^\dagger + \left(\beta B\right)^\dagger		\tag{Linearity of conjugation}
\\ &= \alpha^* A^\dagger + \beta^* B^\dagger	
\\ &= \alpha A^\dagger + \beta B^\dagger	\tag{$\alpha, \beta \in \mathbb{R}$}
\\ &= \alpha A + \beta B	\tag{$A, B$ are Hermitian} \qedhere
\end{align*}

\subsubsection{}

\begin{align*}
\left(\star\right) &= \left(\alpha - \beta\right)\braket{v}{u}
\\ &= \alpha\braket{v}{u} - \beta\braket{v}{u}
\\ &= \alpha\braket{u}{v} - \beta\braket{v}{u}		\tag{$\braket{v}{u}=\braket{u}{v}$}
\\ &= \bra{u}A\ket{v} - \beta\braket{v}{u}			\tag{$A\ket{v}=\alpha\ket{v}$}
\\ &= \bra{u}A\ket{v} - \bra{v}A\ket{u}				\tag{$A\ket{u}=\beta\ket{u}$}
\\ &= 0
\end{align*}

Because $\alpha \neq \beta$ and $\left(\star\right) = 0$, we must conclude that $\braket{v}{u} = 0 \qed$.

\section{}

\subsection{}

For Hermitian matrix $\alpha_i = \alpha_i^\dagger$, $\alpha_i\alpha_i^\dagger = \alpha_i^\dagger\alpha_i = I \iff \alpha_i^2 = I$. 

\begin{itemize}

\item $\alpha_x\cdot\alpha_x = \begin{pmatrix} 0 & 1 \\ 1 & 0 \end{pmatrix}^2 = I$
\item $\alpha_y\cdot\alpha_y = \begin{pmatrix} 0 & -i \\ i & 0 \end{pmatrix}^2 = I$
\item $\alpha_z\cdot\alpha_z = \begin{pmatrix} 1 & 0 \\ 0 & -1 \end{pmatrix}^2 = I$
\item $H \cdot H = \left(\frac{1}{\sqrt{2}}\begin{pmatrix} 1 & 1 \\ 1 & -1 \end{pmatrix}\right)^2 = I$

\end{itemize}

\subsection{} \label{question 4.2}


Denote the i-th column of $U$ by $U_i$. It holds that $U^\dagger U = I$, so from the definition of matrices multiplication we get: 
$$(U_i)^\dagger U_j = I_{ij}$$ 
By definition, 
$$I_{ij} = \delta_{ij} = \begin{cases} 1 & i=j \\ 0 & otherwise \end{cases}$$
Putting it all together:
$$(U_i)^\dagger U_j = \left\langle U_i, U_j\right\rangle = \delta_{ij}$$

\subsection{}

TODO - Orthonormality of column $\to$ unitary matrix

\subsection{}

Let $\lambda$ be an eigenvalue with eigenvector $v\neq \vec{0}$ of unitary matrix $U$. Then,
$$
\left\langle v,v \right\rangle = \left\langle Uv,Uv \right\rangle
	= \left\langle \lambda v, \lambda v \right\rangle =  \lambda^\star\lambda \left\langle v,v \right\rangle \Rightarrow \lambda^\star\lambda = 1 \Rightarrow |\lambda | = 1
$$
\section{}

\subsection{}

\subsubsection{}
%5.1.1
$$
(\vec{a} + \vec{c}) \otimes \vec{b}
\
=
\begin{pmatrix} 
(a_1 + c_1) \vec{b} \\ (a_2 + c_2) \vec{b} \\ \vdots \\ (a_k + c_k) \vec{b} 
\end{pmatrix}
\
=
\begin{pmatrix} 
a_1 \vec{b} \\ a_2 \vec{b} \\ \vdots \\ a_k \vec{b} 
\end{pmatrix}
+
\begin{pmatrix} 
c_1 \vec{b} \\ c_2 \vec{b} \\ \vdots \\ c_k \vec{b} 
\end{pmatrix}
\
=
\vec{a}\otimes\vec{b} + \vec{c}\otimes\vec{b}
$$


\subsubsection{}
%5.1.2
$$
\vec{a} \otimes (\vec{b} + \vec{c})
\
=
\begin{pmatrix}
a_1 (\vec{b} + \vec{c}) \\ a_2 (\vec{b} + \vec{c}) \\ \vdots \\ a_k (\vec{b} + \vec{c})
\end{pmatrix}
\
=
\begin{pmatrix}
a_1 \vec{b}  \\ a_2 \vec{b} \\ \vdots \\ a_k \vec{b}
\end{pmatrix}
+
\begin{pmatrix}
a_1 \vec{c}  \\ a_2 \vec{c} \\ \vdots \\ a_k \vec{c}
\end{pmatrix}
\
=
\vec{a}\otimes\vec{b}+\vec{a}\otimes\vec{c}
$$

\subsubsection{}
%5.1.3 scalar multiplication
$$
\vec{a}\otimes\left(c\cdot\vec{b}\right)
\
=
\begin{pmatrix}
a_1 \cdot c\vec{b}\\ a_2 \cdot c\vec{b}\\ \vdots\\ a_k\cdot c\vec{b}
\end{pmatrix}
\
=
\underbrace{
	\begin{pmatrix}
	c\cdot a_1\vec{b}\\ c\cdot a_2 \vec{b}\\ \vdots\\ c\cdot a_k\vec{b}
	\end{pmatrix}
}_{
	\left(c\cdot \vec{a}\right) \otimes \vec{b}
}
\
=
c\cdot 
\begin{pmatrix}
a_1\vec{b}\\ a_2 \vec{b}\\ \vdots\\ a_k\vec{b}
\end{pmatrix}
\
=
c\cdot\left(\vec{a}\otimes\vec{b}\right)
$$

\subsubsection{}
%5.1.4
Let 
$$
\vec{a} = \begin{pmatrix} a_1 \\ a_2 \end{pmatrix},
\vec{b} = \begin{pmatrix} b_1 \\ b_2 \end{pmatrix},
\vec{c} = \begin{pmatrix} c_1 \\ c_2 \end{pmatrix},
\vec{d} = \begin{pmatrix} d_1 \\ d_2 \end{pmatrix}
\in \mathbb{C}^2
$$ 

We can compute the following tensor products:


\begin{align*} 
\vec{a}\otimes\vec{b} &= \begin{pmatrix} a_1b_1\\a_1b_2\\a_2b_1\\a_2b_2 \end{pmatrix} \\
\vec{c}\otimes\vec{d} &= \begin{pmatrix} c_1d_1\\c_1d_2\\c_2d_1\\c_2d_2 \end{pmatrix}
\end{align*}

Plugging them into the inner product yields:

\begin{align*}
\left\langle\vec{a}\otimes\vec{b},\vec{c}\otimes\vec{d}\right\rangle
=
\begin{pmatrix} a_1b_1\\a_1b_2\\a_2b_1\\a_2b_2 \end{pmatrix}^\dagger
\cdot \begin{pmatrix} c_1d_1\\c_1d_2\\c_2d_1\\c_2d_2 \end{pmatrix}
&=
a_1^\star b_1^\star c_1 d_1
+ a_1^\star b_2^\star c_1 d_2
+ a_2^\star b_1^\star c_2 d_1
+ a_2^\star b_2^\star c_2 d_2 \\
&=
(a_1^\star c_1)(b_1^\star d_1)
+ (a_1^\star c_1)(b_2^\star d_2)
+ (a_2^\star c_2)(b_1^\star d_1)
+ (a_2^\star c_2)(b_2^\star d_2) \\
&=
(a_1^\star c_1 + a_2^\star c_2) (b_1^\star d_1 + b_2^\star d_2)\\
&= \left\langle \vec{a},\vec{c} \right\rangle 
	\left\langle \vec{b},\vec{d} \right\rangle \qed
\end{align*}

\subsection{}

$$
\left(\vec{a}\otimes\vec{b}\right)\otimes\vec{c}
=
\begin{pmatrix} a_1b_1 \\ a_1b_2 \\ a_2b_1 \\ a_2b_2 \end{pmatrix}
\otimes
\begin{pmatrix} c_1 \\ c_2 \end{pmatrix}
=
\begin{pmatrix} a_1b_1c_1 \\ a_1b_1c_2 \\ a_1b_2c_1 \\ a_1b_2c_2 
		\\ a_2b_1c_1 \\ a_2b_1c_2 \\ a_2b_2c_1 \\ a_2b_2c_2 \end{pmatrix}
=
\begin{pmatrix} a_1 \\ a_2 \end{pmatrix}
\otimes
\begin{pmatrix} b_1c_1 \\ b_1c_2 \\ b_2c_1 \\ b_2c_2 \end{pmatrix}
=
\vec{a} \otimes \left( \vec{b} \otimes \vec{c}\right)
$$

\subsection{}

We prove only for matrices of size $2\times2$. Let:
\begin{align*}
	A &= 
	\begin{pmatrix} a_{11} & a_{12} \\ a_{21} & a_{22} \end{pmatrix} \\
	B &=
	\begin{pmatrix} b_{11} & b_{12} \\ b_{21} & b_{22} \end{pmatrix}
\end{align*}

Now we can "explicitly" compute:

\begin{align*}
tr\left( A \otimes B \right) &= tr
\begin{pmatrix}
a_{11}b_{11} & a_{11}b_{12} & a_{12}b_{11} & a_{12}b_{12} \\
a_{11}b_{21} & a_{11}b_{22} & a_{12}b_{21} & a_{12}b_{22} \\
a_{21}b_{11} & a_{21}b_{12} & a_{22}b_{11} & a_{22}b_{12} \\
a_{21}b_{21} & a_{21}b_{22} & a_{22}b_{21} & a_{22}b_{22}
\end{pmatrix} \\
&= a_{11}b_{11} + a_{11}b_{22} + a_{22}b_{11} + a_{22}b_{22} \\
&= \left(a_{11} + a_{22}\right) \left(b_{11} + b_{22}\right) \\
&= tr\left(A\right) tr\left(B\right)
\end{align*}
\subsection{}

As we've seen in previous exercises, for 2 matrices of size $2\times2$ we get
$$
A \otimes B =
\begin{pmatrix}
a_{11}b_{11} & a_{11}b_{12} & a_{12}b_{11} & a_{12}b_{12} \\
a_{11}b_{21} & a_{11}b_{22} & a_{12}b_{21} & a_{12}b_{22} \\
a_{21}b_{11} & a_{21}b_{12} & a_{22}b_{11} & a_{22}b_{12} \\
a_{21}b_{21} & a_{21}b_{22} & a_{22}b_{21} & a_{22}b_{22}
\end{pmatrix}
$$

Denote the $i$-th column in this product by $D_1$. Without loss of generality, we show that $\left\langle D_1, D_2 \right\rangle = 1$ and that $\left\langle D_1, D_1 \right\rangle = 0$. The proof for the other cases is very similar.

\begin{align*}
\left\langle D_1, D_2 \right\rangle &= 
a_{11}^2 b_{11}b_{12} 
+ a_{11}^2 b_{21}b_{22}
+ a_{21}^2 b_{11}b_{12}
+ a_{21}^2 b_{21}b_{22} \\
&= 
a_{11}^2\left(b_{11}b_{12} + b_{21}b_{22}\right) 
+ a_{21}^2\left(b_{11}b_{12} + b_{21}b_{22}\right) \\
&= \left(a_{11}^2 + a_{21}^2\right) 
\underbrace{\left\langle B_1, B_2 \right\rangle}_{=0} = 0 
\\
\\
\left\langle D_1, D_1 \right\rangle &= 
a_{11}^2 b_{11}^2
+ a_{11}^2 b_{21}^2
+ a_{21}^2 b_{11}^2
+ a_{21}^2 b_{21}^2 \\
&=
\left(a_{11}^2 + a_{21}^2\right) \left(b_{11}^2 + b_{21}^2\right) \\
&=
\left\langle A_1, A_1 \right\rangle \left\langle B_1, B_1 \right\rangle = 1
\end{align*}

\subsection{}

Recall our previous notation:
$$
A \otimes B =
\begin{pmatrix}
a_{11}b_{11} & a_{11}b_{12} & a_{12}b_{11} & a_{12}b_{12} \\
a_{11}b_{21} & a_{11}b_{22} & a_{12}b_{21} & a_{12}b_{22} \\
a_{21}b_{11} & a_{21}b_{12} & a_{22}b_{11} & a_{22}b_{12} \\
a_{21}b_{21} & a_{21}b_{22} & a_{22}b_{21} & a_{22}b_{22}
\end{pmatrix}
$$
Also,
$$
\vec{u} \otimes \vec{v} = 
\begin{pmatrix} u_1 v_1 \\ u_1 v_2 \\ u_2 v_1 \\ u_2 v_2 \end{pmatrix}
$$
Thus we get:
\begin{align*}
\left(A \otimes B\right) \left(\vec{u} \otimes \vec{v}\right)
&=
\begin{pmatrix}
a_{11}b_{11}u_1v_1 + a_{11}b_{12}u_1v_2 + a_{12}b_{11}u_2v_1 + a_{12}b_{12}u_2v_2\\
a_{11}b_{21}u_1v_1 + a_{11}b_{22}u_1v_2 + a_{12}b_{21}u_2v_1 + a_{12}b_{22}u_2v_2\\
a_{21}b_{11}u_1v_1 + a_{21}b_{12}u_1v_2 + a_{22}b_{11}u_2v_1 + a_{22}b_{12}u_2v_2\\
a_{21}b_{21}u_1v_1 + a_{21}b_{22}u_1v_2 + a_{22}b_{21}u_2v_1 + a_{22}b_{22}u_2v_2
\end{pmatrix}\\
&= 
\begin{pmatrix}
(a_{11}u_1 + a_{12}u_2)(b_{11}v_1 + b_{12}v_2) \\
(a_{11}u_1 + a_{12}u_2)(b_{21}v_1 + b_{22}v_2)	\\
(a_{21}u_1 + a_{22}u_2)(b_{11}v_1 + b_{12}v_2) \\
(a_{21}u_1 + a_{22}u_2)(b_{21}v_1 + b_{22}v_2)
\end{pmatrix} \\
&=
\begin{pmatrix}
a_{11}u_1 + a_{12}u_2 \\
a_{21}u_1 + a_{22}u_2
\end{pmatrix}
\otimes
\begin{pmatrix}
b_{11}v_1 + b_{12}v_2 \\
b_{21}v_1 + b_{22}v_2
\end{pmatrix} \\
&= A\vec{u} \otimes B\vec{v}
\end{align*}
\end{document}




































