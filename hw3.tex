\documentclass[a4paper,10pt]{hw}
\usepackage[utf8]{inputenc}
\usepackage{amsmath,amsthm,amssymb}
\usepackage[table]{xcolor}
\usepackage{tabularx}
\usepackage{mathtools}

\newtheorem{theorem}{Theorem}[section]
\newtheorem{corollary}{Corollary}[theorem]
\newtheorem{lemma}[theorem]{Lemma}

% Bra-Ket hack from stack exchange
\DeclarePairedDelimiter\bra{\langle}{\rvert}
\DeclarePairedDelimiter\ket{\lvert}{\rangle}
\DeclarePairedDelimiterX\braket[2]{\langle}{\rangle}{#1 \delimsize\vert #2}

% TabularX streched-centered column (Z)
\newcolumntype{Z}{>{\vfill\centering\let\newline\\\arraybackslash\hspace{0pt}}X}

%opening
\title{Introduction to Quantum Information Processing}
\author{Roei Rosenzweig 313590937,\\ Roey Maor 205798440}

\begin{document}

\maketitle

\section{Harmonic Quantum Oscillator - Solution}

\begin{enumerate}

% Question 1 sub section 1
\item
\begin{flalign*}
a + a^\dagger = 2\cdot X\cdot\sqrt\frac{m\omega}{2\hbar} \Rightarrow X = \sqrt{\frac{2\hbar}{m\omega}} \frac{a + a^\dagger}{2} &&& \\
a - a^\dagger = 2\cdot \frac{i}{m\omega}P\cdot\sqrt\frac{m\omega}{2\hbar} \Rightarrow P = \sqrt{\frac{2\hbar}{m\omega}} \frac{m\omega}{i} \frac{a - a^\dagger}{2} &&&
\end{flalign*}

% Question 1 sub section 2
\item

We begin by calculating an expression for the operator $aa^\dagger$, by using $\left[X, P\right] = i\hbar$:

\begin{align*}
aa^\dagger &= \frac{m\omega}{2\hbar} \left(X^2 + \frac{i}{m\omega} PX - \frac{i}{m\omega} XP + \frac{1}{m^2 \omega^2} P^2\right)&& \\
&= \frac{m\omega}{2\hbar} \left(X^2 -i\hbar \frac{i}{m\omega} + \frac{1}{m^2 \omega^2} P^2 \right) &&\\
&= \frac{m\omega}{2\hbar} \left(X^2 + \frac{1}{m^2 \omega^2} P^2 \right) + \frac{m\omega}{2\hbar} \frac{h}{m\omega} = C + \frac{1}{2}
\end{align*}

From symmetry, we get that $a^\dagger a = C - \frac{1}{2} = N$, thus $aa^\dagger = N + 1$. Now we can calculate the hamiltonian:

\begin{align*}
{\cal{H}}  &= \frac{P^2}{2m} + \frac{1}{2}m\omega^2 X^2 = \frac{1}{2m} \left( \sqrt{\frac{2\hbar}{m\omega}} \frac{m\omega}{i} \frac{a - a^\dagger}{2} \right)^2 + \frac{1}{2} m\omega^2 \left( \sqrt{\frac{2\hbar}{m\omega}} \frac{a + a^\dagger}{2}\right)^2 && \\
&= \frac{1}{2m} \frac{2\hbar}{m\omega} \frac{m^2\omega^2}{-1} \frac{a^2 -(N + 1) - N + \left(a^\dagger\right)^2}{4}
+
\frac{1}{2}m\omega^2 \frac{2\hbar}{m\omega} \frac{a^2 + (N + 1) + N +\left(a^\dagger\right)^2}{4} && \\
&= -\omega \hbar \frac{a^2 - (N + 1) - N + \left(a^\dagger\right)^2}{4} + \omega \hbar \frac{a^2 + (N + 1) + N + \left(a^\dagger\right)^2}{4} && \\
&= \hbar\omega \frac{4N + 2}{4} = \hbar\omega \left(N + \frac{1}{2}\right) &&
\end{align*}



% Question 1 sub section 3
\item
for $n>0$:

\begin{align*}
N\ket{n} = a^\dagger a \ket{n} =  \sqrt{n} a^\dagger \ket{n-1} = \sqrt{n}^2 \ket{n} = n\cdot \ket{n}
\end{align*}

for $n=0$:

\begin{align*}
a^\dagger \ket{0} &=& \sqrt{1}\ket{1} \\
a \ket{1} &=& \sqrt{1} \ket{0}\\
\Downarrow \\
&& a a^\dagger \ket{0} = a\ket{1} = \ket{0} = (N + 1)\ket{0} \Rightarrow N\ket{0} = 0\cdot\ket{0}
\end{align*}


% Question 1 sub section 4
\item

Proof by induction. we claim that $\ket{n} = \frac{1}{\sqrt{n!}} \cdot \left( a^\dagger \right) ^n \cdot \ket{0}$ for $n \geq 0$.

	\begin{itemize}
	
	\item base
	for $n=0$ we get $\ket{0} = \frac{1}{\sqrt{0!}} \cdot \left( a^\dagger \right)^0 \cdot \ket{0} = \ket{0}$. 
	
	\item step
	assume $\ket{n} = \frac{1}{\sqrt{n!}} \cdot \left( a^\dagger \right) ^n \cdot \ket{0}$ for some $n \geq 0$:
	
	\begin{align*}
	\ket{n+1} &= \frac{1}{\sqrt{n+1}} a^\dagger \ket{n} = \frac{1}{\sqrt{n+1}} a^\dagger\cdot  \frac{1}{\sqrt{n!}} \cdot \left( a^\dagger \right) ^n \cdot \ket{0} = \frac{1}{\sqrt{n+1!}} \cdot \left( a^\dagger \right) ^{n+1} \cdot \ket{0}
	\end{align*}
	
	\end{itemize}

\end{enumerate}

\section{State Classification - Solution}

\begin{enumerate}

% Question 2 sub section 1
\item

First, lets compute:

$$
\frac{\ket{00} + \ket{++} + \ket{--}}{\alpha} = \frac{1}{a}
\left( 
\begin{pmatrix}
0 \\ 0 \\ 0 \\ 1
\end{pmatrix}
+
\begin{pmatrix}
1 \\ 1 \\ 1 \\ 1
\end{pmatrix}
+
\begin{pmatrix}
1 \\ -1 \\ -1 \\ 1
\end{pmatrix}
\right)
=
\frac{1}{\alpha}
\begin{pmatrix}
2 \\ 0 \\ 0 \\ 3
\end{pmatrix}
$$

This state is pure (by definition - Its a vector in Hilbert space), and it is \textbf{entangled} - it can't be written as a tensor product of 2 vectors. Separable states must have the property $\alpha_{01}\alpha{10}=\alpha{00}\alpha{11}$, and in our case $0\cdot 0 \neq 3\cdot 2$.
The norm of a pure state must be 1, and we can use this to calculate $\alpha$:

\begin{align*}
\frac{9}{\alpha^2} + \frac{4}{\alpha^2} &= 1 \\
\alpha &= 13
\end{align*}

% Question 2 subsection 2
\item

\end{enumerate}

\end{document}




































